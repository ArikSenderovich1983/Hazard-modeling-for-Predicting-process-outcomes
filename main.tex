\documentclass[runningheads]{llncs}
\usepackage{amsmath, amssymb, amsfonts}
\usepackage{graphicx}
\usepackage{booktabs}
\usepackage{algorithm}
\usepackage{algpseudocode} % for \Require, \State, \For, etc.
\usepackage{xcolor}
\usepackage{hyperref}

\begin{document}

\title{Process-Aware Prediction of Treatment-Resistant Depression: Integrating Dynamic Pathway Mining with Survival Modeling for Early Intervention}
\author{Zeinab Soleimani \and Arik Senderovich  \and Mathias Weske}
%\institute{York University, University of Potsdam, Hasso Plattner Institute, Mount Sinai}
\maketitle

% ============================================================
\begin{abstract}
Treatment-Resistant Depression (TRD) develops in a substantial fraction of patients diagnosed with Major Depressive Disorder (MDD) and remains one of psychiatry's greatest therapeutic challenges. Prior work has characterized TRD and non-TRD pharmacotherapy pathways using process mining, but predictive modeling of TRD onset remains underexplored. We propose a process-aware predictive framework that integrates event-log–based process mining with survival analysis to estimate the risk of TRD onset in real time. Three methodological variants are compared: (A) process-derived dynamic covariates embedded in survival models; (B) sequence-to-outcome deep models operating directly on event logs; and (C) hybrid cluster–survival approaches that combine process abstraction with time-to-event modeling. Experiments on large-scale OMOP-formatted EHR data show that early trajectory dynamics strongly predict later TRD development, enabling earlier and more targeted intervention.
\keywords{Process Mining \and Survival Analysis \and Treatment-Resistant Depression \and Dynamic Prediction \and Healthcare Analytics}
\end{abstract}

% ============================================================
\section{Introduction}
Major Depressive Disorder (MDD) affects over 280 million individuals worldwide and imposes high clinical and economic burden. Approximately 20--30\% of patients develop Treatment-Resistant Depression (TRD), defined as insufficient response to at least two adequate antidepressant trials. Detecting treatment resistance early could substantially improve patient outcomes and optimize resource allocation.

In healthcare analytics, process mining has emerged as a powerful method for reconstructing real-world care pathways from event logs. Yet, most applications are descriptive—mapping what happens—rather than predictive—estimating what will happen. Outcome prediction in process-driven environments requires bridging process mining with statistical learning, especially when events unfold over irregular time intervals, as in psychiatric pharmacotherapy.

This paper addresses the following question:
\begin{quote}
\emph{Can early antidepressant treatment trajectories, represented as process patterns, predict the onset of treatment resistance (TRD) before it becomes clinically manifest?}
\end{quote}
We propose a unified framework combining process mining with survival analysis to forecast TRD onset dynamically.

% ============================================================
\section{Background}
\subsection{Process Mining for Clinical Pathways}
Briefly review process mining fundamentals: event logs, trace variants, Directly-Follows Graphs (DFGs), abstraction levels, and how these capture patient-level care sequences. Discuss prior descriptive analyses of TRD vs. non-TRD trajectories.

\subsection{Survival and Multi-State Modeling}
Outline standard and recurrent-event survival models, Cox proportional hazards, and interval censoring. Highlight the challenge that ``non-TRD'' is not absence of resistance but right-censoring.

\subsection{Process-Aware Outcome Prediction}
Position the work within the broader field of process outcome prediction and process-aware survival modeling. Mention applications in manufacturing, patient flow, and business process management.

% ============================================================
\section{Proposed Model and Framework}
We represent each patient as a sequence of antidepressant exposures over time, encoded as an event log $L = \{ \langle e_{1}, e_{2}, ..., e_{n} \rangle_i \}_{i=1}^{N}$, where each event $e_j = (a_j, t_j, c_j)$ comprises the activity (drug category), timestamp, and clinical context.

The goal is to estimate the hazard $h_i(t)$ or probability $P(\text{TRD onset} | L_i^{\le t})$ given the partial history.

Three complementary modeling options are considered.

% ------------------------------------------------------------
\subsection{Option A: Process-Derived Dynamic Covariates (PM+Survival)}
Process mining metrics (e.g., transition frequencies, pathway entropy, switch rates) are extracted within sliding time windows and used as time-varying covariates in a Cox or joint survival model.

\begin{algorithm}[h]
\caption{Process-Aware Dynamic Survival Modeling}
\begin{algorithmic}[1]
\Require Event log $L$, time horizon $T$, window size $w$
\For{each patient $i$}
    \For{each time window $t=1,\ldots,T/w$}
        \State Extract process features $X_i(t)$ from events in $[t-w, t]$
        \State Update survival model: $h_i(t)=h_0(t)\exp(\beta^\top X_i(t))$
    \EndFor
\EndFor
\State Estimate TRD risk curves via partial likelihood or landmarking.
\end{algorithmic}
\end{algorithm}

% ------------------------------------------------------------
\subsection{Option B: Sequence-to-Outcome Deep Model}
Learn representations directly from event sequences using RNNs or Transformers that embed both temporal and categorical event data. The model predicts survival probability or discrete-time risk.

\begin{algorithm}[h]
\caption{Sequence-to-Outcome Deep Learning}
\begin{algorithmic}[1]
\Require Tokenized event sequences $\{L_i\}$ with timestamps
\State Embed events into vector space $(a_j, \Delta t_j)$
\State Train model $f_\theta$ to output $P_\theta(\text{TRD at } t|L_i^{\le t})$
\State Loss = negative log partial likelihood or time-dependent cross-entropy
\State Evaluate with C-index, time-dependent AUC
\end{algorithmic}
\end{algorithm}

% ------------------------------------------------------------
\subsection{Option C: Hybrid Process Clustering + Survival}
Cluster early trajectories using process-mining similarity (e.g., trace distance, DFG embeddings). Fit separate survival models per cluster to quantify TRD risk conditional on pathway type.

\begin{algorithm}[h]
\caption{Hybrid Process Clustering and Survival Modeling}
\begin{algorithmic}[1]
\Require Event log $L$, clustering function $\Phi(\cdot)$
\State Compute pairwise trace distances $d(L_i,L_j)$
\State Derive clusters $C_1,\ldots,C_k = \Phi(L,d)$
\For{each cluster $C_k$}
    \State Fit survival model: $h_k(t)=h_{0k}(t)\exp(\beta_k^\top X_i)$
\EndFor
\State Predict risk per patient by cluster membership.
\end{algorithmic}
\end{algorithm}

% ============================================================
\section{Experimental Setup}
\subsection{Data and Cohort}
Describe the EHR dataset (OMOP format), inclusion criteria, TRD definition, censoring, and preprocessing.

\subsection{Research Question}
Evaluate: \emph{How early can we predict TRD onset using partial process histories?}
Compare models (A–C) on prediction horizon (3, 6, 12 months after MDD diagnosis).

\subsection{Metrics}
Use discrimination (C-index), calibration (Brier score), and net benefit (Decision Curve Analysis). Assess interpretability through feature importance and transition-level hazard ratios.

\subsection{Implementation}
Detail PM4Py for event abstraction, \texttt{lifelines}/\texttt{scikit-survival} for modeling, and PyTorch for sequence models. Mention reproducibility via open repository.

% ============================================================
\section{Results and Discussion}
Summarize expected findings:
\begin{itemize}
  \item Dynamic process covariates significantly improve early prediction of TRD.
  \item Deep models capture complex switching behavior but trade off interpretability.
  \item Cluster-based survival highlights distinct high-risk pathway archetypes.
\end{itemize}
Discuss clinical implications for adaptive intervention and potential integration into decision-support systems.

% ============================================================
\section{Related Work}
Cite prior work in:
\begin{itemize}
  \item Process mining in psychiatry and pharmacotherapy (Soleimani et al. 2025)
  \item Process outcome prediction (Teinemaa, Camargo, Van der Aalst)
  \item Survival analysis for recurrent or multi-state events in healthcare
  \item Hybrid PM–ML frameworks (Leemans, Mazhar, Weske)
\end{itemize}

% ============================================================
\section{Conclusion}
We proposed a process-aware predictive framework for early detection of Treatment-Resistant Depression, combining descriptive process mining with time-to-event modeling. By treating process features as dynamic covariates or embedding full event sequences into predictive architectures, we enable data-driven early-warning systems that bridge interpretability and accuracy. Future work includes multi-site validation, incorporation of psychotherapy and demographic covariates, and integration into clinical workflows for adaptive antidepressant management.

\bibliographystyle{splncs04}
\bibliography{refs}

\end{document}
